\documentclass[11pt]{article}
\usepackage{listings}
\usepackage{tikz}
\usepackage{alltt}
\usepackage{hyperref}
\usepackage{url}
%\usepackage{algorithm2e}
\usetikzlibrary{arrows,automata,shapes}
\tikzstyle{block} = [rectangle, draw, fill=blue!20, 
    text width=5em, text centered, rounded corners, minimum height=2em]
\tikzstyle{bt} = [rectangle, draw, fill=blue!20, 
    text width=1em, text centered, rounded corners, minimum height=2em]

\newtheorem{defn}{Definition}
\newtheorem{crit}{Criterion}
\newcommand{\true}{\mbox{\sf true}}
\newcommand{\false}{\mbox{\sf false}}

\newcommand{\handout}[5]{
  \noindent
  \begin{center}
  \framebox{
    \vbox{
      \hbox to 5.78in { {\bf Software Testing, Quality Assurance and Maintenance } \hfill #2 }
      \vspace{4mm}
      \hbox to 5.78in { {\Large \hfill #5  \hfill} }
      \vspace{2mm}
      \hbox to 5.78in { {\em #3 \hfill #4} }
    }
  }
  \end{center}
  \vspace*{4mm}
}

\newcommand{\lecture}[4]{\handout{#1}{#2}{#3}{#4}{Lecture #1}}
\topmargin 0pt
\advance \topmargin by -\headheight
\advance \topmargin by -\headsep
\textheight 8.9in
\oddsidemargin 0pt
\evensidemargin \oddsidemargin
\marginparwidth 0.5in
\textwidth 6.5in

\parindent 0in
\parskip 1.5ex
%\renewcommand{\baselinestretch}{1.25}

\usepackage{enumitem}

\newtheorem{prop}{Proposition}
\newtheorem{lemma}{Lemma}
\usepackage{ebproof}
\newcommand{\qedsymbol}{\rule{1ex}{1ex}}

\lstset{ %
language=Java,
basicstyle=\ttfamily,commentstyle=\scriptsize\itshape,showstringspaces=false,breaklines=true,numbers=left}

%\usepackage{fontspec}
%\setmonofont{Cousine}[Scale=MatchLowercase]

\begin{document}

\lecture{5c --- January 28, 2025}{Winter 2025}{Patrick Lam}{version 2}

Here's a proof of semantic equivalence between $s_1 = \mathsf{skip}\:;\:S$
and $s_2 = S$. From page 11 of the L05 notes, we have:

\[ s_1 \mbox{ and } s_2 \mbox{ are semantically equivalent if } \forall q.~ \langle s_1, q \rangle \Downarrow q_1 \mbox{ and } \langle s_2, q \rangle \Downarrow q_2 \mbox{ implies } q_1 = q_2. \]

So, if we take any $q$, and let $q'$ satisfy $\langle S, q \rangle \Downarrow q'$ for that $q'$, then we have to show that $\langle \mathsf{skip} \texttt{\:;\:} S, q \rangle \Downarrow q'$ also. (If $q'$ doesn't exist, because $S$ doesn't terminate, then there is no proof obligation).

The composition rule says:
\[
  \begin{prooftree}
    \hypo{\langle s_1, q \rangle \Downarrow q''}
    \hypo{\langle s_2, q'' \rangle \Downarrow q'}
  \infer2{~~\langle s_1 \texttt{\:;\:} s_2, q \rangle \Downarrow q'}
  \end{prooftree}  \hspace*{4em} \
  \]
and the skip rule says:
  \[
  \begin{prooftree}
  \infer0{~~\langle \textsf{skip}, q \rangle \Downarrow q}
  \end{prooftree} 
  \]

Applying the rule for \textsf{skip} gives us $\langle \mathsf{skip}, q \rangle \Downarrow q$.
We observe that the premises for the composition rule hold, if we substitute $s_1 = \mathsf{skip}$, $q'' = q$, and $s_2 = S$.
We thus have the conclusion
\[
\langle \mathsf{skip} \texttt{\:;\:} S, q \rangle \Downarrow q',
\]
as required.
\end{document}
