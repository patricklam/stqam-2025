\documentclass[11pt]{article}
\usepackage{listings}
\usepackage{tikz}
\usepackage{alltt}
\usepackage{hyperref}
\usepackage{url}
%\usepackage{algorithm2e}
\usetikzlibrary{arrows,automata,shapes}
\tikzstyle{block} = [rectangle, draw, fill=blue!20, 
    text width=5em, text centered, rounded corners, minimum height=2em]
\tikzstyle{bt} = [rectangle, draw, fill=blue!20, 
    text width=1em, text centered, rounded corners, minimum height=2em]

\newtheorem{defn}{Definition}
\newtheorem{crit}{Criterion}
\newcommand{\true}{\mbox{\sf true}}
\newcommand{\false}{\mbox{\sf false}}

\newcommand{\handout}[5]{
  \noindent
  \begin{center}
  \framebox{
    \vbox{
      \hbox to 5.78in { {\bf Software Testing, Quality Assurance and Maintenance } \hfill #2 }
      \vspace{4mm}
      \hbox to 5.78in { {\Large \hfill #5  \hfill} }
      \vspace{2mm}
      \hbox to 5.78in { {\em #3 \hfill #4} }
    }
  }
  \end{center}
  \vspace*{4mm}
}

\newcommand{\lecture}[4]{\handout{#1}{#2}{#3}{#4}{Lecture #1}}
\topmargin 0pt
\advance \topmargin by -\headheight
\advance \topmargin by -\headsep
\textheight 8.9in
\oddsidemargin 0pt
\evensidemargin \oddsidemargin
\marginparwidth 0.5in
\textwidth 6.5in

\parindent 0in
\parskip 1.5ex
%\renewcommand{\baselinestretch}{1.25}

\usepackage{enumitem}

\newtheorem{prop}{Proposition}
\newtheorem{lemma}{Lemma}
\usepackage{ebproof}
\newcommand{\qedsymbol}{\rule{1ex}{1ex}}

\lstset{ %
language=Java,
basicstyle=\ttfamily,commentstyle=\scriptsize\itshape,showstringspaces=false,breaklines=true,numbers=left}

%\usepackage{fontspec}
%\setmonofont{Cousine}[Scale=MatchLowercase]

\begin{document}

\lecture{5b --- January 22, 2025}{Winter 2025}{Patrick Lam}{version 1}

I said I'd do a structural induction example in class since there isn't one in the notes yet.

For a structure built from a grammar, we would use the grammar to provide the proof obligations for
the structural induction. WHILE has terminals, arithmetic expressions, Boolean expressions, and
statements.

\begin{prop} All WHILE programs that do not contain any \texttt{while} statements always terminate.
\end{prop}

\paragraph{Proof.} We are going to use the big-step semantics to prove this. In particular,
we are going to show that any \texttt{while}-less program is going to admit a finite derivation
that finishes with some final state.

A WHILE program is an \texttt{slist}, which is one or more statements. Statements may contain
arithmetic or boolean expressions.

\begin{lemma}
  Evaluation of any boolean or arithmetic expression always yields a value and terminates.
\end{lemma}

\paragraph{Proof of lemma.} We first consider arithmetic expressions. The base case
for these expressions are integers $n$ and variables $x$. The semantics contains rules:

\[
\begin{array}{ccc}
  \begin{prooftree}
  \infer0{\langle n, q \rangle \Downarrow n},
\end{prooftree} & \textrm{ and~~ }
\begin{prooftree}
  \infer0{\langle x, q \rangle \Downarrow q(x)}
\end{prooftree}
\end{array}.
\]
which clearly yield values and terminate. Next, we have the three grammar rules
that build arithmetic expressions from smaller ones: negation, parentheses, and arithmetic operators.
For each of those, we assume that all smaller expressions yield values and terminate. I'll show the
proof for the arithmetic operator $+$. (I went and sneakily removed division\ldots).
\[
  \begin{prooftree}
    \hypo{\langle e_1, q \rangle \Downarrow n_1}
    \hypo{\langle e_2, q \rangle \Downarrow n_2}
  \infer2{\langle e_1 + e_2, q \rangle \Downarrow n_1 + n_2}
  \end{prooftree}
  .
\]
We assume, by induction, that $e_1$ and $e_2$ have the desired property. This derivation tree shows that if you build
an arithmetic expression with $+$, then it also has the desired property. It would be OK to say ``similarly for
- and *''. You should probably show at least one unary operator case too.
\[
  \begin{prooftree}
    \hypo{\langle e, q \rangle \Downarrow n}
  \infer1{\langle (e), q \rangle \Downarrow n},
  \end{prooftree}
  .
\]
Here we say that because $e$ has the property, this derivation shows that $(e)$ also yields a value and terminates.
Unary negation is the same.

For boolean expressions, the base cases are \texttt{true} and \texttt{false}, and you can quote the inference
rules
\[
\begin{array}{ccc}
  \begin{prooftree}
  \infer0{\langle \texttt{true}, q \rangle \Downarrow \texttt{true}},
\end{prooftree} & \textrm{ and~~ }
\begin{prooftree}
  \infer0{\langle \texttt{false}, q \rangle \Downarrow \texttt{false}}
\end{prooftree}
\end{array}.
\]
The relational operators rely on us having shown termination for arithmetic expressions.
\[
  \begin{prooftree}
    \hypo{\langle e_1, q \rangle \Downarrow n_1}
    \hypo{\langle e_2, q \rangle \Downarrow n_2}
  \infer2{\langle e_1 < e_2, q \rangle \Downarrow (n_1 < n_2)}
  \end{prooftree}
  .
\]
That is, we know that $e_1$ and $e_2$ evaluate to integers, and then we apply the rule and return
\texttt{true} if $n_1 < n_2$ and \texttt{false} otherwise.
Finally, the boolean operators rely on the induction hypothesis.
\[
  \begin{prooftree}
    \hypo{\langle e_1, q \rangle \Downarrow t_1}
    \hypo{\langle e_2, q \rangle \Downarrow t_2}
  \infer2{\langle t_1 \texttt{~and~} t_2, q \rangle \Downarrow (t_1 \wedge t_2)}
  \end{prooftree}
  .
\]
We assume that $e_1$ and $e_2$ evaluate to boolean values $t_1$ and $t_2$ and then we have a rule which
yields a value for $t_1$ and $t_2$.~~\qedsymbol

Returning to the proof for statements, then, we have a number of statements for which we need to prove the claim.
The base cases for this proof are \texttt{skip}, assignment statements, and the \texttt{print\_state},
\texttt{assert}, \texttt{assume}, and \texttt{havoc} statements. We only gave semantics for \texttt{skip} and
assignment, so we'll not talk about the other statements either.

\[
  \begin{prooftree}
    \hypo{\langle e, q \rangle \Downarrow n}
  \infer1{\langle x := e, q \rangle \Downarrow q[x:=n]}
  \end{prooftree}
  .
\]

This is a base case for this proof because there is no statement in
the hypothesis. We rely on the lemma to show that we can evaluate $e$
to value $n$ and terminate. This rule shows that we can evaluate an
assignment statement and terminate. (No more yielding a value here in
the big-step semantics; evaluation just changes the state.)

The inductive cases are the statement list \texttt{slist} and the if statement. I guess I can show them both.
For if, we have termination because we inductively assume termination for the then and else clauses.

\[
  \begin{prooftree}
    \hypo{\langle s_1, q \rangle \Downarrow q'}
    \hypo{\langle e, q \rangle \Downarrow \mbox{true}}
  \infer2{\langle \texttt{if $e$ then $s_1$ else $s_2$}, q \rangle \Downarrow q'}
  \end{prooftree}
  .
  \]
  This is the then clause. You can say ``similiarly for \texttt{else}''. We have $s_1$ terminates by induction
  and so the \texttt{if} also terminates by this derivation.

Finally, for the statement list:  
\[
  \begin{prooftree}
    \hypo{\langle s_1, q \rangle \Downarrow q^1}
    \hypo{\cdots}
    \hypo{\langle s_n, q^{n-1} \rangle \Downarrow q^n}
  \infer3{\langle \{ s_1 ; \cdot ;s s_n \}, q \rangle \Downarrow q^n}
  \end{prooftree}
  .
  \]
Because, inductively, all of the $s_i$ terminate, then so does the list of $s$. ~~\qedsymbol

\end{document}
