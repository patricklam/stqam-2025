\documentclass[11pt]{article}
\usepackage{mathpartir}
\usepackage{pifont}
\usepackage{multicol}
\usepackage{listings}
\usepackage{pgf}
\usepackage{tikz}
\usepackage{alltt}
\usepackage{hyperref}
\usepackage{url}
\usepackage{amsmath}
\usepackage{amssymb}
\usepackage{dafny}
\usetikzlibrary{arrows,automata,shapes,positioning}
\tikzstyle{block} = [rectangle, draw, fill=blue!20, 
    text width=5em, text centered, rounded corners, minimum height=2em]
\tikzstyle{bt} = [rectangle, draw, fill=blue!20, 
    text width=1em, text centered, rounded corners, minimum height=2em]
\newcommand{\xmark}{\ding{55}}

\newtheorem{defn}{Definition}
\newtheorem{crit}{Criterion}
\newcommand{\true}{\mbox{\sf true}}
\newcommand{\false}{\mbox{\sf false}}

\newcommand*\circled[1]{\tikz[baseline=(char.base)]{
            \node[shape=circle,draw,inner sep=2pt] (char) {#1};}}


\newcommand{\handout}[5]{
  \noindent
  \begin{center}
  \framebox{
    \vbox{
      \hbox to 5.78in { {\bf Software Testing, Quality Assurance and Maintenance } \hfill #2 }
      \vspace{4mm}
      \hbox to 5.78in { {\Large \hfill #5  \hfill} }
      \vspace{2mm}
      \hbox to 5.78in { {\em #3 \hfill #4} }
    }
  }
  \end{center}
  \vspace*{4mm}
}

\newcommand{\lecture}[4]{\handout{#1}{#2}{#3}{#4}{Lecture #1}}
\topmargin 0pt
\advance \topmargin by -\headheight
\advance \topmargin by -\headsep
\textheight 8.9in
\oddsidemargin 0pt
\evensidemargin \oddsidemargin
\marginparwidth 0.5in
\textwidth 6.5in

\parindent 0in
\parskip 1.5ex
%\renewcommand{\baselinestretch}{1.25}

\usepackage{enumitem}

\newtheorem{prop}{Proposition}
\newtheorem{lemma}{Lemma}
\usepackage{ebproof}
\newcommand{\qedsymbol}{\rule{1ex}{1ex}}
\newcommand{\sem}[3]{\langle #1, #2 \rangle \Downarrow #3}

\lstset{ %
language=Java,
basicstyle=\ttfamily,commentstyle=\scriptsize\itshape,showstringspaces=false,breaklines=true,numbers=left}

%\usepackage{fontspec}
%\setmonofont{Cousine}[Scale=MatchLowercase]

\begin{document}

\lecture{14 --- March 10, 2025}{Winter 2025}{Patrick Lam}{version 2}

We motivated Dafny last time by talking about Cedar. In brief, Dafny
proves user-specified annotations and verifies the absence of runtime
errors---the things that are usually undefined behaviour or crashes,
like out of bounds array indexes, null dereferences, etc.  We've seen
how to use symbolic execution to avoid these as well, but Dafny is a
fully-static approach that aims to work on real code (written in the
Dafny language).

Other applications of verified software besides Cedar:
\begin{itemize}[noitemsep]
\item Paris Metro line 14 control system (B)\footnote{\url{https://www.clearsy.com/wp-content/uploads/2020/03/Formal-methods-for-Railways-brochure-mai-2020.pdf}}: 110,000 lines of models, 86,000 lines of Ada.
\item seL4 (Haskell, Isabelle/HOL, C)\footnote{\url{https://sel4.systems/Info/FAQ/proof.html}}: the seL4 microkernel conforms to its specification and maintains integrity and confidentiality.
\item CompCert (Coq)\footnote{https://inria.hal.science/hal-01238879}: a formally verified optimizing compiler
\item IronClad (Dafny)\footnote{https://www.microsoft.com/en-us/research/project/ironclad/}: guarantees about how remote apps behave
\end{itemize}

The first three applications were carried out by verification experts; the last one was done by systems programmers. In all of these cases, the systems were designed as verified systems from the start.

\section*{Dafny}

I don't know how long it will take, but in this set of lecture notes, we'll aim to get through
the first online guide, which shows how to verify basic imperative code (including loops but not advanced
topics like objects, sequences and sets, data structures, lemmas). I'm going to summarize what's
there, and in class, we'll work on exercises together.

\begin{itemize}[noitemsep]
\item \url{https://dafny.org/dafny/OnlineTutorial/guide}
\item \url{https://dafny.org/blog/2023/12/15/teaching-program-verification-in-dafny-at-amazon/}
\end{itemize}

\section*{Dafny: annotations}
This annotation in Dafny should be familiar to you from what we've seen in class:
\begin{lstlisting}[language=dafny]
  forall k: int :: 0 <= k < a.Length ==> 0 < a[k]
\end{lstlisting}
or in ASCII,
\begin{verbatim}
  forall k: int :: 0 <= k < a.Length ==> 0 < a[k]
\end{verbatim}
It says that all elements of array $a$ are positive.

Let's start Dafny and write a method that ensures this as a precondition.
Can you fill this in on your computer, experimenting with how to make it verify?
\begin{lstlisting}[language=dafny]
  method all_positive() returns (rv:array<int>)
  ensures forall k: int :: 0 <= k < rv.Length ==> 0 < rv[k]
  {
    var arr := new int[2];
    // ...
    return arr;
  }  
\end{lstlisting}

\section*{Dafny: methods}
We saw a method above. A \emph{method} is a piece of imperative, executable code.
It's like a Java method, or a procedure or function in other languages. Dafny uses
``function'' to mean something else. Here's another method declaration:
\begin{lstlisting}[language=dafny]
  method Abs(x:int) returns (y:int)
  {
    // ...
  }
\end{lstlisting}
Dafny methods can return multiple values:
\begin{lstlisting}[language=dafny]
  method MultipleReturns(x:int, y:int) returns (more:int, less:int)
  {
    more := x + y;
    less := x - y;
  }
\end{lstlisting}
Note the use of ``:='' for assignment (and ``=='' for equality).

Return variables are just like local variables and can be assigned multiple times.
Parameters are read-only.

Let's fill in the body of Abs.
\begin{lstlisting}[language=dafny]
  method Abs(x:int) returns (y:int)
  {
    if x < 0 { // must write the {
        return -x;
      } else {
        return x;
     }
  }
\end{lstlisting}

\section*{Preconditions and postconditions}
What's special about Dafny, of course, is that you can write \textsf{requires}
and \textsf{ensures} clauses on methods, and Dafny will try to verify them, or
complain. Let's add a postcondition to the Abs method.
\begin{lstlisting}[language=dafny]
method AbsWithPostcondition(x:int) returns (y:int)
  ensures 0 < y
{
  if x < 0 {
    return -x;
  } else {
    return x;
  }
}
\end{lstlisting}
Note that we use the name of the return value, \textsf{y}. Now, why doesn't this verify?

We can write multiple postconditions as well.
\begin{lstlisting}[language=dafny]
  method MultipleReturns(x:int, y:int) returns (more:int, less:int)
    ensures less < x
    ensures x < more
  {
    more := x + y;
    less := x - y;
  }
\end{lstlisting}
OK, what goes wrong here? How can we verify this method?

This doesn't help, but we can also write:
\begin{lstlisting}[language=dafny]
  method MultipleReturns(x:int, y:int) returns (more:int, less:int)
    ensures less < x < more
  {
    more := x + y;
    less := x - y;
  }
\end{lstlisting}

There are two reasons why Dafny can't prove something.
(1) it's ``too hard'' for Dafny and Boogie/Z3 to prove; (2)
it's actually not true. Test cases can help with (2).
The other thing you will need, in general, is invariants.

So, of course \textsf{less $<$ x} isn't true if \textsf{y} is negative.
Let's make sure that it's impossible to call \textsf{MultipleReturns}
in that case.
\begin{lstlisting}[language=dafny]
  method MultipleReturns(x:int, y:int) returns (more:int, less:int)
    requires y >= 0
    ensures less < x < more
  {
    more := x + y;
    less := x - y;
  }
\end{lstlisting}
You can also write more than one requires clause. OK, here's an exercise.

\paragraph{Exercise 0.} Write a method \textsf{Max} that takes two
integer parameters and returns their maximum. Add appropriate annotations
and make sure your code verifies.

\begin{lstlisting}[language=dafny]
  method Max(a: int, b: int) returns (c: int)
    // write a postcondition
  {
    // write code
  }
\end{lstlisting}

\section*{Assertions}
We can put assertions anywhere, for instance:
\begin{lstlisting}[language=dafny]
  method TryingOutAssertions() {
    assert 2 < 3;
    // what about asserting something not true?
  }
\end{lstlisting}
Assertions are especially helpful for debugging Dafny, because it tries to prove that they hold
on all executions. So if you think that Dafny knows something, write it as an assertion and see
if it does or not.

Local variables are useful, including for assertions.
\begin{lstlisting}[language=dafny]
  method m()
  {
    var x, y, z: bool := 1, 2, true;
  }
\end{lstlisting}
Dafny infers types for local variables, or you can explicitly specify them. Using variables:
\begin{lstlisting}[language=dafny]
  method TestAbsWithPostcondition(){
    var v := AbsWithPostcondition(3);
    assert 0 <= v;
  }
\end{lstlisting}

\paragraph{Exercise 1.} Write a test method that calls your \textsf{Max} method from Exercise 0
and then asserts something about the result.

But what about:
\begin{lstlisting}[language=dafny]
  method AbsWithPostcondition(x:int) returns (y:int)
  ensures 0 <= y
  {
    if x < 0 {
      return -x;
    } else {
      return x;
    }
  }

  method TestAbs()
  {
    var v := AbsWithPostcondition(3);
    assert 0 <= v;
    assert v == 3; // oops, can't prove this
  }
\end{lstlisting}
Looking at \textsf{AbsWithPostcondition}, clearly \textsf{v} is 3 if we use the body of that method.
But Dafny doesn't know that. It just looks at the postcondition. From the postcondition, it knows
that \textsf{v} is non-negative, but not exactly what it is.

Indeed, Dafny doesn't know the difference between \textsf{AbsWithPostcondition} and:
\begin{lstlisting}[language=dafny]
  method NotAbs(x:int) returns (y:int)
  ensures 0 <= y
  {
    return 5;
  }
\end{lstlisting}
and it also can't prove this:
\begin{lstlisting}[language=dafny]
  method TestNotAbs(x:int) returns (y:int)
  ensures 0 <= y
  {
    var v := NotAbs(3);
    assert v == 3; // actually not true
  }
\end{lstlisting}
So what postcondition do we want? How about this?
\begin{lstlisting}[language=dafny]
  method AbsBetterPostcondition(x:int) returns (y:int)
  ensures 0 <= y
  ensures 0 <= x ==> y == x
  {
    if x < 0 {
      return -x;
    } else {
      return x;
    }
  }
\end{lstlisting}
Well\ldots
\begin{lstlisting}[language=dafny]
  method TestAbsBetter(x:int) returns (y:int)
  {
    var v := AbsBetterPostcondition(5);
    assert v == 5;
    var w := AbsBetterPostcondition(-2);
    assert w == 2;
  }
\end{lstlisting}
OK, one way to write the postcondition we really want is:
\begin{lstlisting}[language=dafny]
  method AbsFullPostcondition(x:int) returns (y:int)
  ensures 0 <= y
  ensures 0 <= x ==> y == x
  ensures x < 0 ==> y == -x
  {
    if x < 0 {
      return -x;
    } else {
      return x;
    }
  }
\end{lstlisting}
Or we can write:
\begin{lstlisting}[language=dafny]
  ensures 0 <= y && (y == x || y == -x)
\end{lstlisting}
There can be more than one way to write a postcondition.

Now, we want to eliminate the redundancy between the postcondition and the implementation.
Functions can help here. But, first, some exercises.

\paragraph{Exercise 2.} Using a precondition, change \textsf{Abs} such that it can only be called
on negative values. Simplify the body of \textsf{Abs} into just one return statement and make sure
the method still verifies.

\paragraph{Exercise 3.} Keeping the postconditions of \textsf{Abs} the same as for
\textsf{AbsFullPostcondition}, change the body of the method to just \textsf{y := x + 2}. What
precondition do you need to impose on the method so that it verifies? What about body
\textsf{y := x + 1}? What does that precondition say about when you can call the method?

% requires x == -1
% requires false

\section*{Functions}

Here is a function. A function body must consist of exactly one expression with the correct type.
\begin{lstlisting}[language=dafny]
  function abs(x:int) : int
  {
    if x < 0 then -x else x
  }
\end{lstlisting}

Why functions? They can be used directly in specifications (e.g. asserts, requires, ensures). Here, look.
\begin{lstlisting}[language=dafny]
  method m()
  {
    assert abs(3) == 3;
  }
\end{lstlisting}
We don't need to store the result of function \textsf{abs} in a temporary local variable, nor did we have
to write a postcondition for it. Dafny doesn't forget about the function body. It is allowed to write
preconditions and postconditions for functions, but not required.

\paragraph{Exercise 4.} Write a function \textsf{max} that returns the larger of two integer parameters.
Write a test method using an \textsf{assert} that checks that your function is correct (at least on one
case).

\paragraph{Exercise 5.} Change the postcondition of method \textsf{Abs} to use function \textsf{abs}, make sure
that \textsf{Abs} still verifies, and then change the body of \textsf{Abs} to also use the function.

Let's continue using functions. Here is a na\"ive implementation of the Fibonacci function.

\begin{lstlisting}[language=dafny]
function fib(n: nat): nat
{
  if n == 0 then 0
  else if n == 1 then 1
  else fib(n-1) + fib(n-2)
}
\end{lstlisting}

Some notes. (1) We had \textsf{int}s before, but now we have
\textsf{nat}s, i.e. natural numbers, which can't be negative; these
are a subset type of \textsf{int}. (By the way, Dafny \textsf{int}s
and \textsf{nat}s are unbounded.) (2) You wouldn't actually want to
\emph{calculate} Fibonacci numbers this way, but it obviously matches
the definition, and we can ask Dafny to prove that a (faster) implementation
also computes the same thing as this function.

\begin{lstlisting}[language=dafny]
method ComputeFib(n: nat) returns (b: nat)
  ensures b == fib(n)
{
  // ...
}
\end{lstlisting}

\section*{Loops and Loop Invariants}
The usual more efficient way to compute Fibonacci numbers is using a loop. We've talked
about loops in the context of Hoare logic, and Dafny is similar, except that it does a lot of
proving for you. But it doesn't supply invariants.

Here's a loop, with an invariant.
\begin{lstlisting}[language=dafny]
method FirstLoop(n:nat)
{
  var i := 0;
  while i < n
    invariant 0 <= i
  {
    i := i + 1;
  }
  assert i == n;
}
\end{lstlisting}
We can see that $\mathsf{i == n}$ at the end of the loop, but does Dafny know that? We can ask it, using
\textsf{assert}. (It doesn't). We need to strengthen the loop invariant. If we try $0 \leq i < n$, then
Dafny complains that the invariant might not hold on entry and it might not be maintained by the loop body.
Why is that?

Well, in any case, if we use the invariant $i \leq i \leq n$, then Dafny is satisfied with everything.

\paragraph{Exercise 6.} Change the loop invariant to $0 \leq i \leq n + 2$. Does the loop still verify?
Does the assertion $i == n$ after the loop still verify?

\paragraph{Exercise 7.} With the original loop invariant, change the loop guard from $i < n$ to
$i \neq n$. Do the loop and assertion after the loop still verify? Why or why not?

Let's implement the Fibonacci method.
\begin{lstlisting}[language=dafny]
method ComputeFib(n: nat) returns (b: nat)
  ensures b == fib(n)
{
  var i := 1;
  var a := 0;
  b := 1;
  while i < n
  {
    a, b := b, a + b;
    i := i + 1;
  }
}
\end{lstlisting}
(Note that Dafny allows assignment to \textsf{a} and \textsf{b} simultaneously).

We have postcondition $b = \mathsf{fib}(n)$, and we also know
that $i == n$, so it must be that $b == \mathsf{fib}(i)$ which we can have as a loop invariant. This
is promising as part of the loop invariant: it talks about the loop counter. Hence, we have what we had before,
plus this.
\begin{lstlisting}[language=dafny]
  invariant 0 <= i <= n
  invariant b == fib(i)
\end{lstlisting}
We also know that $a$ is the previous Fibonacci number. So,
\begin{lstlisting}[language=dafny]
  invariant a == fib(i - 1)
\end{lstlisting}
Finally, since Dafny \textsf{nat} starts at 0, we include:
\begin{lstlisting}[language=dafny]
  if n == 0 { return 0; }
\end{lstlisting}
All together:
\begin{lstlisting}[language=dafny]
function fib(n:nat): nat
{
  if n == 0 then 0
  else if n == 1 then 1
  else fib(n-1) + fib(n-2)
}

method ComputeFib(n:nat) returns (b:nat)
  ensures b == fib(n)
{
  if n == 0 { return 0; }
  var i := 1;
  var a := 0;
  b := 1;
  while i < n
    invariant 0 < i <= n
    invariant a == fib(i - 1)
    invariant b == fib(i)
  {
    a, b := b, a + b;
    i := i + 1;
  }
}
\end{lstlisting}

% stopped about here after 90minutes

\paragraph{Exercise 8.} It is possible to simplify \textsf{ComputeFib}.
Write a simpler program by not introducing \textsf{a} as the Fibonacci number
that precedes \textsf{b}, but instead introducing \textsf{c} as the variable that
succeeds \textsf{b}. Verify that your program is correct according to the mathematical
definition of Fibonacci.

(I'm omitting Exercise 9 from the tutorial because it's not really that useful.)

As you know, an issue with just verifying partial correctness is that it doesn't tell you
anything at all if the program doesn't terminate. When you run the program, you can see it not
terminating, but if you're just proving it, you might write non-terminating code.

\section*{Termination}
Dafny does prove termination for loops and recursion. It does so by proving that there is something
that decreases and that is bounded. Usually Dafny manages to guess what is decreasing. When that is
something over natural numbers, the lower bound is often 0. You can hover over
the loop to see the inferred decreases clause. Sometimes Dafny can't guess, and you have to provide an annotation.

\begin{lstlisting}[language=dafny]
method Decreases()
{
  var i := 20;
  while 0 < i
    invariant 0 <= i
    decreases i // could be inferred
    {
      i := i - 1;
    }
}
\end{lstlisting}
Note that the thing that is decreasing might be the negation of something that's increasing.
\begin{lstlisting}[language=dafny]
method Decreases2()
{
  var i, n := 0, 20;
  while i < n
    invariant 0 <= i <= n
    decreases n - i
    {
      i := i + 1;
    }
}
\end{lstlisting}
Dafny can guess this one too, from the loop condition $i < n$.
The lower bound is from $i \leq n$ in the invariant, which implies
$n - i \geq 0$. The bound doesn't have to be constant, as in binary search.

\paragraph{Exercise 10.} In the loop above, the invariant $i \leq n$
and the negation of the loop guard allow us to conclude $i == n$ after the loop,
as we've previously checked with an \textsf{assert}. If the loop guard was instead
$i \neq n$ (as in Exercise 7), then the negation of the guard immediately gives $i == n$,
without needing the loop invariant. Change the loop guard to $i \neq n$ and delete the invariant.
What happens?

We have also seen the recursive function \textsf{fib}. Dafny also guesses the decreases
annotation here. We could specify it explicitly:

\begin{lstlisting}[language=dafny]
function fib(n:nat): nat
  decreases n
{
  if n == 0 then 0
  else if n == 1 then 1
  else fib(n-1) + fib(n-2)
}
\end{lstlisting}

There is a longer Dafny tutorial on termination, and we'll probably go through that one as well.

\section*{Arrays}
In Dafny we have one-dimensional arrays $\mathsf{array}\langle T \rangle$ for any $T$ (though here we'll only consider
$\mathsf{array}\langle int \rangle$), as well as $\mathsf{array}?\langle T \rangle$ which includes possibly-null arrays.

Array access is as you might expect, $a[i]$, and there is a built-in
length field: $\mathsf{a}.\mathsf{Length}$.  Part of Dafny's
compile-time verification includes showing that all array accesses are
in-bounds, i.e.  betweeen 0 and $\textsf{Length}-1$ inclusive.

We saw array allocation right at the beginning of this document:
\begin{lstlisting}[language=dafny]
    var arr := new int[2];
\end{lstlisting}

Let's express the easy part of the contract for the \textsf{Find} method.
\begin{lstlisting}[language=dafny]
method FindPartialContract(a: array<int>, key:int) returns (index:int)
  // partial contract
  ensures 0 <= index ==> index < a.Length && a[index] == key
{
  // can you write code that satisfies this postcondition?
  // it can be done with one statement.
}
\end{lstlisting}
This is the case where the key is in the array. It's at index \textsf{index}.
The conjunct $\mathsf{index} < a.\mathsf{Length}$ guards against out-of-bounds array
accesses that might be caused by \textsf{a[index]}, and works because Dafny has short-circuit
logical ands/ors (like C and many other languages).

For the other case, we need to say that \textsf{key} does not exist at \emph{any}
index in the array, so we need quantifiers.

\paragraph{Quantifiers.} Dafny supports quantifiers in expressions. Use a double-colon (::)
after the variable name; it shows up as a dark $\cdot$ in Emacs and VS Code.

\begin{lstlisting}[language=dafny]
method WithQuantifier()
{
  assert forall k :: k < k + 1;
}
\end{lstlisting}
(This verifies, but there's a warning about triggers. We won't talk about this here.)

Typically we quantify over some more limited set than ``all integers'' (Dafny infers the type of
$k$, though you can specify it if you want). For instance, all integers that are valid indices into an array,
i.e. between 0 and $a.\mathsf{Length}$.
\begin{lstlisting}[language=dafny]
  forall k :: 0 <= k <= a.Length => a[k] != key
\end{lstlisting}
We can thus write the whole contract:
\begin{lstlisting}[language=dafny]
method Find(a: array<int>, key:int) returns (index:int)
  ensures 0 <= index ==> index < a.Length && a[index] == key
  ensures index < 0 ==> forall k :: 0 <= k < a.Length ==> a[k] != key
{
    // TODO implement this
}
\end{lstlisting}
Linear search is the easiest way to implement \textsf{Find}:
\begin{lstlisting}[language=dafny]
  index := 0;
  while index < a.Length {
    if a[index] == key {
      return;
    }
    index := index + 1;
  }
  index := -1;
\end{lstlisting}
and when we plug that into the contract, we see that Dafny can't prove the postcondition
saying that if \textsf{index} is negative, then \textsf{key} is not in the array. Makes sense:
all we know upon leaving the loop is that \textsf{index} has reached the end of the array.
We need to add an invariant showing that the part of the array we've visited so far does not
contain \textsf{key}.
\begin{lstlisting}[language=dafny]
  invariant forall k :: 0 <= k < index ==> a[k] != key
\end{lstlisting}
Dafny complains: it doesn't know that \textsf{index} doesn't go off the end of the array.
This invariant works.
\begin{lstlisting}[language=dafny]
  invariant 0 <= index <= a.Length
\end{lstlisting}
We often see loop invariants where we build the property that we need one element at a time,
as we do here. In this case, the property is that the visited part of the array doesn't contain
the key that we're looking for.

\paragraph{Exercise 11.} Write a method that takes an integer array, which it requires to have
at least one element, and returns an index to the maximum of the array's elements. Annotate
the method with pre- and postconditions that state the intent of the method, and annotate
its body with a loop invariant to verify it.
\begin{lstlisting}[language=dafny]
method FindMax(a: array<int>) returns (index:int)
  // TODO put preconditions and postconditions here  
{
  // TODO implement this
}
\end{lstlisting}

Linear search is fine, but if the array is sorted, then binary search is faster. We can't
prove that binary search is correct unless we can somehow express a property about the array
being sorted. Which we can. But let's see if we can refactor this property, using predicates.

\paragraph{Predicates.} Dafny has functions, which can return anything. Dafny also has
predicates, which return booleans. We are going to define the \textsf{sorted} predicate
which will return true if and only iff its array argument is sorted. Using this predicate
makes our expressions shorter and more readable (as long as the name of the predicate
corresponds to its meaning).

\begin{lstlisting}[language=dafny]
predicate sorted(a: array<int>)
{
  forall j, k :: 0 <= j < k < a.Length ==> a[j] <= a[k]
}
\end{lstlisting}
A predicate is like a function, but the return type must be boolean, and so you don't have to write this.

Dafny won't compile this, and says that there is an insufficient \textsf{reads} clause.

\paragraph{Framing.} Dafny needs to know when heap locations might change.

We know that functions and predicates can't change the heap. But
Dafny isn't a purely functional language---methods can change the heap.

So, Dafny wants to know when the value of \textsf{sorted} might change
between calls. In a world where there are two arrays, and a method
writes to one of the arrays, a \textsf{reads} annotation tells
Dafny \emph{which} array \textsf{sorted} is reading from, and whether
that array might have been changed by the method. If not, then the
array has the same sortedness before and after the method was called.

\begin{lstlisting}[language=dafny]
predicate sorted(a: array<int>)
  reads a
{
  forall j, k :: 0 <= j < k < a.Length ==> a[j] <= a[k]
}
\end{lstlisting}
The \textsf{reads} annotation specifies a set of memory
locations that the function is allowed to access. Here, it is
an array, which means that \textsf{sorted} can access all
of the elements of the array. It could also be object fields and
sets of objects. We'll probably check that later.

As we've seen, Dafny checks that the \textsf{reads} clause
specifies everything that the function might read. This includes
any functions that the function transitively calls.

Conversely, as we've seen, methods can read anything they want.
But they must declare any changes to memory using a \textsf{modifies}
clause.

These clauses allow Dafny to reason about methods and functions
in isolation, and hence they make verification much more tractable.

Dafny only requires framing information for memory on the heap,
which includes arrays and objects, but not sets, sequences, or multisets
(which we'll talk about later).

\paragraph{Exercise 12.} Create a \textsf{sortedAndDistinct}
predicate that returns true exactly when the array is sorted and all its
elements are distinct.

\paragraph{Exercise 13.} Create a \textsf{sortedAndNonNull}
predicate that accepts references to a possibly-null array and that
returns true exactly when the array is sorted and not null.

\paragraph{Binary Search.} Predicates: software engineering principles for
annotations. Here we use the \textsf{sorted} predicate in the precondition
for \textsf{BinarySearch}. Dafny thus knows that the array is sorted
when verifying the method body.

\begin{lstlisting}[language=dafny]
method BinarySearch(a: array<int>, value: int) returns (index:int)
  requires 0 <= a.Length && sorted(a)
  ensures 0 <= index ==> index < a.Length && a[index] == value
  ensures index < 0 ==> forall k :: 0 <= k < a.Length ==> a[k] != value
{
    // ...
}
\end{lstlisting}

Here's an implementation of binary search.
\begin{lstlisting}[language=dafny]
  var low, high := 0, a.Length;
  while low < high
    invariant 0 <= low <= high <= a.Length
    invariant forall i ::
      0 <= i < a.Length && !(low <= i < high) ==> a[i] != value
    {   
      var mid := (low + high) / 2;
      if a[mid] < value {
        low := mid + 1;
      } else if value < a[mid] {
        high := mid;
      } else {
        return mid;
      }
    }
    return -1;
\end{lstlisting}
Let's look at the implementation and the invariants.
\begin{lstlisting}[language=dafny]
    invariant 0 <= low <= high <= a.Length
    invariant forall i ::
      0 <= i < a.Length && !(low <= i < high) ==> a[i] != value
\end{lstlisting}
The first invariant says that the search is looking between
$[\mathsf{low}, \mathsf{high})$, and the second invariant
says that the searched-for value is not anywhere outside
that range.

Reasoning about how the first invariant continues to hold is
straightforward.

For the second invariant, Dafny needs to show that changing
\textsf{low} or \textsf{high} doesn't miss the value that we are
searching for. It can use the fact that predicate \textsf{sorted}
holds. Taking the first \textsf{if} case, we are bumping up
\textsf{low} to above \textsf{mid}, based on sortedness and the fact
that $\mathsf{a}[\mathsf{mid}] < \mathsf{value}$, which ensure that
the value isn't in that part of the range.

Off-by-one errors are easy to write. Dafny catches them in this case.

\paragraph{Exercise 14.} Change the assignments in the body of
\textsf{BinarySearch} to set \textsf{low} to \textsf{mid} or
to set \textsf{high} to $\mathsf{mid}-1$. What goes wrong?

\section*{Conclusion}
We've seen a bunch of the major features of Dafny and used it to verify
some simple code, up to a binary search implementation. There are more
features: objects, sequences and sets, data structures, lemmas, modules, etc.
We'll probably talk about them in the next few lectures. You don't
really need them to do the assignment this year. Maybe next year.

There is a lot of information about Dafny online, including other
tutorials and the reference.

SE 465 is not actually the course on specifications. That's SE 463.
But we use formal specifications in this part of SE 465 to prove
that implementations conform to these specifications, and that
they are using other code consistently with that code's specifications.
Both tests and verification are tools that you can use to verify
that your code does what it's supposed to do. Tests are more
practical, while verification is conceptually more powerful.

\end{document}
