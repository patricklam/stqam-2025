\documentclass[t]{beamer}

\usepackage{etex}
\usepackage{amsmath}
\usepackage{extarrows} % write symbols over arrows
\usepackage{stmaryrd}  % llbracket and rrbracket symbols
\usepackage{amsfonts}  % for \mathbb
\usepackage{amssymb}   % For \triangleq
\usepackage{proof}     % for \infer
\usepackage{verbatim}
\usepackage{latexsym}
\usepackage{color}
\usepackage{float}
\usepackage{url}
\usepackage{graphicx}
\usepackage{pifont}
\usepackage{textpos}
\usepackage{xcolor,colortbl}
\usepackage{tikz}                       % to draw CFGs
\usepackage{xspace}
\usetikzlibrary{arrows,shapes,calc,positioning}
\usetikzlibrary{calc,shapes.callouts,shapes.arrows,shapes.geometric,fit}
\usepackage{listings}
% \usepackage{showexpl}
\usepackage{mathpartir}

%\usepackage[defblank]{paralist}
\usetheme{Madrid}
\usecolortheme{seahorse}

% Colors
\definecolor{apricot}{RGB}{250,190,160}
\definecolor{codebg}{RGB}{214,215,250}
\definecolor{olive}{rgb}{0.3, 0.4, .1}
\definecolor{fore}{RGB}{249,242,215}
\definecolor{back}{RGB}{51,51,51}
\definecolor{title}{RGB}{255,0,90}
\definecolor{dgreen}{rgb}{0.,0.6,0.}
\definecolor{gold}{rgb}{1.,0.84,0.}
\definecolor{JungleGreen}{cmyk}{0.99,0,0.52,0}
\definecolor{BlueGreen}{cmyk}{0.85,0,0.33,0}
\definecolor{RawSienna}{cmyk}{0,0.72,1,0.45}
\definecolor{Magenta}{cmyk}{0,1,0,0}

% shows how to change default (blue) colours in the default beamer theme
% found here: http://joerglenhard.wordpress.com/tag/latex/
\definecolor{UWRed}{RGB}{145,11,46}
\definecolor{UWPurple}{RGB}{190,51,218}
\setbeamercolor{title}{fg=UWPurple}
\setbeamercolor{frametitle}{fg=UWPurple}
\setbeamercolor{structure}{fg=UWPurple}
\setbeamertemplate{navigation symbols}{}
\setbeamertemplate{caption}{\raggedright\insertcaption\par}

% adds logo in the footer
\logo{\includegraphics[scale=.2]{uow_logo.jpg}}
% don't bother me with fulsome warning messages
\tolerance=100000

\mathchardef\mhyphen="2D

\title{Semantics of Symbolic Execution}
\author{Arie Gurfinkel}

\date{\today}

\newcommand{\bZ}{\mathbb{Z}}
\newcommand{\sem}[3]{\langle #1, #2 \rangle \Downarrow #3}
\newcommand{\ssem}[4]{\langle #1, #2, #3 \rangle \Downarrow #4}
\newcommand{\cSkip}{\textbf{skip}}
\newcommand{\cDo}{\textbf{do}}
\newcommand{\cWhile}{\textbf{while}}
\newcommand{\cIf}{\textbf{if}}
\newcommand{\cThen}{\textbf{then}}
\newcommand{\cElse}{\textbf{else}}
\newcommand{\cHavoc}{\textbf{havoc}}
\renewcommand{\gets}{\mathbin{:=}}
\newcommand{\cPrint}{\textbf{print\_state}}
\newcommand{\seq}{\mathbin{;}}
\newcommand{\cAssume}{\textbf{assume}}
\newcommand{\cAssert}{\textbf{assert}}
\newcommand{\vTrue}{\text{true}}
\newcommand{\vFalse}{\text{false}}
\newcommand{\cequiv}{\equiv_{con}}

\begin{document}
\large\frame{\titlepage}

\begin{frame}{Preliminaries}
  \begin{itemize}
  \item $L$ is a set of program variables (locations)
  \item \emph{concrete} state $q$ (a.k.a., an \emph{environment}) is a
    map from program variables $L$ to integers $\bZ$
  \item $Q : L \to \bZ$
    is the set of all states.
  \end{itemize}
  \vfill
  \begin{block}{Notation}
    \begin{itemize}
      \item $q(v)$ stands for the value of variable $v$ in state $q$
      \item $[\;]$ stands for an empty state (no variables)
    \item  $[x \gets u, y \gets v]$ stands for a state s.t. $x$ is $u$
      and $y$ is $v$
    \item $q[x\gets n]$ is a state obtained from $q$ by sub the
      value of $x$ by $n$
    \end{itemize}

  \end{block}
  \vfill
\end{frame}

\begin{frame}[c]{Operational Semantics: Expressions}
  \renewcommand{\arraystretch}{2.5}
  \scalebox{0.9}{%
\begin{tabular}{l@{\hskip 1in}r}
  \inferrule{ }{\sem{n}{q}{n}}{$n\in\bZ$} &
  \inferrule{ }{\sem{v}{q}{q(v)}}{$v \in L$} \\[0.3in]
  \inferrule{\sem{a_1}{q}{n_1}\\\sem{a_2}{q}{n_2}}{\sem{a_1 +
  a_2}{q}{n_1 + n_2}} &
  \inferrule{\sem{a_1}{q}{n_1}\\\sem{a_2}{q}{n_2}}{\sem{a_1 \times
  a_2}{q}{n_1 \times n_2}} \\[0.3in]
  \inferrule{\sem{a_1}{q}{n_1}\\\sem{a_2}{q}{n_2}}{\sem{a_1 <
  a_2}{q}{n_1 < n_2}} &
  \inferrule{\sem{a_1}{q}{n_1}\\\sem{a_2}{q}{n_2}}{\sem{a_1 >
  a_2}{q}{n_1 > n_2}} \\[0.3in]
    \inferrule{\sem{b_1}{q}{r_1}\\\sem{b_2}{q}{r_2}}{\sem{b_1 \land
  b_2}{q}{r_1 \land r_2}} &
    \inferrule{\sem{b_1}{q}{r_1}\\\sem{b_2}{q}{r_2}}{\sem{b_1 \lor
  b_2}{q}{r_1 \lor r_2}}\\[0.3in]
\end{tabular}}
\end{frame}


\begin{frame}[c]{Operational Semantics: Statements}
  \renewcommand{\arraystretch}{1.5}
  \scalebox{0.9}{%
\begin{tabular}{lr}
  \inferrule{ }{\sem{\cSkip}{q}{q}} &
  \inferrule{ }{\sem{\cPrint}{q}{q}} \\[0.3in]
  \inferrule{\sem{s_1}{q}{q''}\\\sem{s_2}{q''}{q'}}{\sem{s_1 \seq
  s_2}{q}{q'}} &
  \inferrule{\sem{e}{q}{n}}{\sem{x \gets e}{q}{q[x\gets n]}} \\[0.3in]
  \inferrule{\sem{b}{q}{\vTrue}\\\sem{s_1}{q}{q'}}{\sem{\cIf\;b\;\cThen\;s_1\;\cElse\;s_2}{q}{q'}} &
  \inferrule{\sem{b}{q}{\vFalse}\\\sem{s_2}{q}{q'}}{\sem{\cIf\;b\;\cThen\;s_1\;\cElse\;s_2}{q}{q'}}\\[0.3in]
  \inferrule{\sem{b}{q}{\vFalse}}{\sem{\cWhile\;b\;\cDo\;s}{q}{q}} &
                                                                     \inferrule{ }{\sem{\cHavoc\;x}{q}{q[x\gets n]}}\\[0.3in]

  \inferrule{\sem{b}{q}{\vTrue}\\\sem{s\seq\cWhile\;b\;\cDo\;s}{q}{q'}}{\sem{\cWhile\;b\;\cDo\;s}{q}{q'}}
\end{tabular}}
\end{frame}

\newcommand{\aststyle}[1]{\textsf{#1}}
\newcommand{\Plus}{\aststyle{Plus}}
\newcommand{\Times}{\aststyle{Times}}
\newcommand{\Lt}{\aststyle{Lt}}
\newcommand{\Gt}{\aststyle{Gt}}
\newcommand{\TAnd}{\aststyle{And}}
\newcommand{\TOr}{\aststyle{Or}}


\begin{frame}{Symbolic States and Path Conditions}
  \begin{itemize}
  \item A \emph{symbolic state} (or symbolic environment) $q$ is a map
    from program variables to \emph{symbolic expressions}
  \item A \emph{path condition} is a formula over symbolic expressions
  \item A \emph{judgment} in symbolic execution has a form
    \[\sem{s}{q, pc}{q',pc'}\] where $s$ is a statement, $q$ and $q'$
    are the input and output symbolic environments, respectively, and
    $pc$ and $pc'$ are input and output path conditions, respectively
  \end{itemize}
  \begin{block}{}
    A path condition $pc'$ is \emph{satisfiable} iff there is a
    concrete input state $c$ and concrete output state $c'$ such
    that $s$ started in state $c$ reaches state $c'$, i.e.,
    $\sem{s}{c}{c'}$
    \end{block}
\end{frame}

\begin{frame}[c]{Symbolic Semantics: Expressions}
  \renewcommand{\arraystretch}{2.5}
  \scalebox{0.9}{%
\begin{tabular}{l@{\hskip 1in}r}
  \inferrule{ }{\sem{n}{q}{n}}{$n\in\bZ$} &
  \inferrule{ }{\sem{v}{q}{q(v)}}{$v \in L$} \\[0.3in]
  \inferrule{\sem{a_1}{q}{n_1}\\\sem{a_2}{q}{n_2}}{\sem{a_1 +
  a_2}{q}{\Plus(n_1,n_2)}} &
  \inferrule{\sem{a_1}{q}{n_1}\\\sem{a_2}{q}{n_2}}{\sem{a_1 \times
  a_2}{q}{\Times(n_1, n_2)}} \\[0.3in]
  \inferrule{\sem{a_1}{q}{n_1}\\\sem{a_2}{q}{n_2}}{\sem{a_1 <
  a_2}{q}{\Lt(n_1, n_2)}} &
  \inferrule{\sem{a_1}{q}{n_1}\\\sem{a_2}{q}{n_2}}{\sem{a_1 >
  a_2}{q}{\Gt(n_1, n_2)}} \\[0.3in]
    \inferrule{\sem{b_1}{q}{r_1}\\\sem{b_2}{q}{r_2}}{\sem{b_1 \land
  b_2}{q}{\TAnd(r_1, r_2)}} &
    \inferrule{\sem{b_1}{q}{r_1}\\\sem{b_2}{q}{r_2}}{\sem{b_1 \lor
  b_2}{q}{\TOr(r_1, r_2)}}\\[0.3in]
\end{tabular}}
\end{frame}

\begin{frame}[c]{Symbolic Semantics: Statements (1/2)}
  \renewcommand{\arraystretch}{1.5}
\begin{tabular}{l}
  \inferrule{ }{\sem{\cSkip}{q,pc}{q,pc}} \\[0.3in]
  \inferrule{ }{\sem{\cPrint}{q,pc}{q,pc}} \\[0.3in]
  \inferrule{\sem{s_1}{q,pc}{q'',pc''}\\\sem{s_2}{q'',pc''}{q',pc'}}{\sem{s_1 \seq s_2}{q,pc}{q',pc'}}\\[0.3in]
  \inferrule{\sem{e}{q}{r}}{\sem{x \gets e}{q,pc}{q[x\gets r],pc}} \\[0.3in]
  \inferrule{ \text{$R$ is a fresh symbolic constant}}{\ssem{\cHavoc\;x}{q}{pc}{q[x\gets R],pc}}
\end{tabular}
\end{frame}

\begin{frame}[c]{Symbolic Semantics: Statements (2/2)}
  \renewcommand{\arraystretch}{1.5}
\begin{tabular}{l}
    \inferrule{\sem{b}{q}{r}\\ pc \land r \text{ is SAT}\\\sem{s_1}{q,pc \land r}{q',pc'}}{\sem{\cIf\;b\;\cThen\;s_1\;\cElse\;s_2}{q,pc}{q',pc'}} \\[0.3in]
  %
  \inferrule{\sem{b}{q}{r}\\ pc \land \neg r \text{ is
  SAT}\\\sem{s_2}{q,pc \land \neg r}{q',pc'}}{\sem{\cIf\;b\;\cThen\;s_1\;\cElse\;s_2}{q,pc}{q',pc'}} \\[0.3in]
  %
  %
  \inferrule{\sem{b}{q}{r}\\pc\land\neg r\text{ is
  SAT}}{\sem{\cWhile\;b\;\cDo\;s}{q,pc}{q,pc\land \neg r}} \\[0.3in]
  \inferrule{\sem{b}{q}{r}\\ pc\land r\text{ is
  SAT}\\\sem{s\seq\cWhile\;b\;\cDo\;s}{q,pc\land r}{q',pc'}}{\sem{\cWhile\;b\;\cDo\;s}{q,pc}{q',pc'}}
                                    \\[0.3in]
\end{tabular}
\end{frame}

\begin{frame}{Concolic Semantics: Notation}
  \begin{itemize}
    \item Program variables $L$ are
      partitioned into symbolic and concrete
      \begin{align*}Sym(L) &= \text{symbolic} & Con(L) &= \text{concrete} \end{align*}\\\vfill
    \item Possibly there are no symbolic (or concrete) variables\\\vfill
    \item $Sym(a)$ and $Con(a)$ indicates whether a given variable $a$ is symbolic or
      concrete, respectively.
  \end{itemize}
\end{frame}

\begin{frame}{Concolic States}
  \begin{itemize}
  \item A concolic state is a triple
    \[q = \langle c, s, pc \rangle\] where $c$ is concrete,
    $s$ is symbolic, and $pc$ is a path condition
    \begin{align*} con(q) &= c &  sym(q) &= s & pc(q)&= pc \end{align*}
\\\vfill
  \item every variable $v$ in $sym(q)$ has a \emph{concrete shadow}
    $con(q)(v)$. That is, concrete state can evaluate all variables
  \end{itemize}
\end{frame}

\begin{frame}{Equivalence and containment}
  \begin{block}{Equivalence of concrete states}
    Two concrete states $c_1$ and $c_2$ are equivalent,
    $c_1 \cequiv c_2$, whenever they agree on concrete variables:
    \[
      c_1 \cequiv c_2 \iff \forall a \in Con(L) \cdot c_1(a) = c_2(a)
    \]
  \end{block}
  \vfill
  \begin{block}{Containment of concrete states}
    $c \models \langle s, pc \rangle$ means that concrete
    $c$ is contained in symbolic $\langle s, pc \rangle$
  \end{block}
\end{frame}

\begin{frame}{Concolic Semantics of Expressions}
  %\setlength{\plitemsep}{50pt}
  \begin{itemize}
  \item The semantics of expressions is as usual with variables
    evaluated based on their kind: concrete variables are evaluated
    over $con(q)$ and symbolic over $sym(q)$:
\[
\begin{array}{ll}
\inferrule{con(a) \\ \sem{a}{con(q)}{v}}{\sem{a}{q}{v}}&
\inferrule{sym(a) \\ \sem{a}{sym(q)}{v}}{\sem{a}{q}{v}}
\end{array}
\]
\\\vfill

\item Expressions that do not depend on symbolic variables are
  evaluated concretely
\\\vfill

\item Expressions that depend on symbolic variables are evaluated
  symbolically (i.e., their value is some AST)
\end{itemize}
\end{frame}

\begin{frame}{Concolic Semantics: Simple Statements}
For most statements, the semantics is extended by applying both
symbolic and concrete operational semantics in parallel:

\vspace{0.3in}
\[
  \inferrule{\sem{s}{con(q)}{c}\\\sem{s}{sym(q),pc(q)}{s',pc'}\\c\models\langle
  s', pc' \rangle}%
  {\sem{s}{q}{\langle c, s', pc' \rangle} }
\]
\vspace{0.3in}

The last pre-condition ensures that the concrete and symbolic states
are chosen consistently.
\end{frame}

\begin{frame}{Concolic Semantics: Assignment}
Assignment of values to concrete variables is limited to concrete
values only:
\vspace{0.2in}
\[
  \inferrule{\sem{e}{q}{n}\\con(x)\\n\in\mathbb{Z}}{\sem{x \gets e}{q}{q[x\gets n]}}
\]
\vspace{0.2in}
\begin{itemize}
\item It is not possible to assign symbolic variables (or symbolic
expressions) to concrete variables!

\item To assign a symbolic value to a concrete variable either
  \begin{itemize}
  \item \hspace{2pt} make concrete variable symbolic (always easy to do), or
  \item \hspace{2pt} make symbolic value concrete (described later on)
  \end{itemize}

\item Assignment to symbolic variables also assigns to their concrete shadows.
\end{itemize}
\end{frame}

\begin{frame}{Concolic Semantics: If-statement}
At if-statement, concolic execution can chose to switch to the branch
that is not consistent with current concrete state, as long as the
concrete state can be adjusted. We only show one of the cases:
\vspace{0.2in}
\[
  \inferrule{\sem{b}{con(q)}{\vTrue}\\
    \sem{b}{q}{r}\\ pc(q) \land \neg r \text{ is
      SAT}\\c \models
    \langle sym(q), pc(q) \land \neg r\rangle\\ c \cequiv
    con(q)\\\sem{s_2}{\langle c, sym(q), pc(q) \land\neg r \rangle}{q'}}%
  {\sem{\cIf\;b\;\cThen\;s_1\;\cElse\;s_2}{q}{q'}}
\]

\end{frame}

\begin{frame}{Concolic Semantics: Concretization}
  \emph{Concretization} turns symbolic variables (or values) into the
  values of their concrete shadows.  The choice of concretization is
  captured in a \emph{concretization constraints} in the path
  condition:
  \vspace{0.2in}
  \[
  \inferrule{ sym(x) \\ \sem{x}{con(q)}{n} \\ \sem{x}{sym(q)}{r}\\\sem{s}{\langle con(q),
      sym(q)[x\gets n], pc(q)\land r=n \rangle}{q'}}%
  {
    \sem{s}{q}{q'}
  }
\]

\vspace{0.2in} That is, if $x$ is a symbolic variable with symbolic
value $r$ and it is currently shadowed concretely by a concrete value
$n$, then its value can be concretized to $n$ as long as the
path condition is updated with $r=n$ to reflect the concretization

\end{frame}

\end{document}
%%% Local Variables:
%%% mode: latex
%%% TeX-master: t
%%% End:
