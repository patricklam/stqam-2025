\documentclass{beamer}

% TODO: print out https://www.fuzzingbook.org/code/Intro_Testing.py

\usetheme{Boadilla}

%\includeonlyframes{current}

\usepackage{times}
\usefonttheme{structurebold}
\usepackage{listings}

\usepackage{pgf}
\usepackage{tikz}
\usepackage{alltt}
\usepackage[normalem]{ulem}
\usetikzlibrary{arrows}
\usetikzlibrary{automata}
\usetikzlibrary{shapes}
\usepackage{amsmath,amssymb}
\usepackage{rotating}
\usepackage{ulem}

\usetikzlibrary{arrows,automata,shapes}
\tikzstyle{block} = [rectangle, draw, fill=blue!20, 
    text width=5em, text centered, rounded corners, minimum height=2em]
\tikzstyle{bt} = [rectangle, draw, fill=blue!20, 
    text width=4em, text centered, rounded corners, minimum height=2em]

\lstdefinelanguage{JavaScript}{
  keywords={typeof, new, true, false, catch, function, return, null, catch, switch, var, if, in, while, 
do, else, case, break},
  keywordstyle=\color{blue}\bfseries,
  ndkeywords={class, export, boolean, throw, implements, import, this},
  ndkeywordstyle=\color{darkgray}\bfseries,
  identifierstyle=\color{black},
  sensitive=false,
  comment=[l]{//},
  morecomment=[s]{/*}{*/},
  commentstyle=\color{purple}\ttfamily,
  stringstyle=\color{red}\ttfamily,
  morestring=[b]',
  morestring=[b]''
}

%\setbeamercovered{dynamic}
\setbeamertemplate{footline}[page number]{}
\setbeamertemplate{navigation symbols}{}
\usefonttheme{structurebold}

\title{Software Testing, Quality Assurance \& Maintenance---Lecture 5b}
\author{Patrick Lam\\University of Waterloo}
\date{January 22, 2025}

\colorlet{redshaded}{red!25!bg}
\colorlet{shaded}{black!25!bg}
\colorlet{shadedshaded}{black!10!bg}
\colorlet{blackshaded}{black!40!bg}

\colorlet{darkred}{red!80!black}
\colorlet{darkblue}{blue!80!black}
\colorlet{darkgreen}{green!80!black}

\newcommand{\rot}[1]{\rotatebox{90}{\mbox{#1}}}
\newcommand{\gray}[1]{\mbox{#1}}

\newenvironment{changemargin}[1]{% 
  \begin{list}{}{% 
    \setlength{\topsep}{0pt}% 
    \setlength{\leftmargin}{#1}% 
    \setlength{\rightmargin}{1em}
    \setlength{\listparindent}{\parindent}% 
    \setlength{\itemindent}{\parindent}% 
    \setlength{\parsep}{\parskip}% 
  }% 
  \item[]}{\end{list}}


\usepackage{ebproof}

\begin{document}

\begin{frame}
  \titlepage
\end{frame}

\part{Structural Induction Example}
\begin{frame}
  \partpage
\end{frame}


\usebackgroundtemplate{}
\begin{frame}
  \frametitle{How Structural Induction Works}
  \Large
  \begin{changemargin}{2em}
    Our structures are defined by grammars.\\[1em]

    Use the grammar to provide proof obligations for structural induction.\\[1em]

    WHILE language: terminals, arithmetic expressions, Boolean expressions, statements.
  \end{changemargin}
\end{frame}

\begin{frame}
  \frametitle{Goal}
  \Large
  \begin{changemargin}{2em}

    \emph{Proposition.} All WHILE programs that do not contain any \texttt{while} statements always terminate.\\[1em]

    Terminate = there is a finite derivation that finishes with some final state.\\[1em]

    We'll use the big-step semantics to prove this.
  \end{changemargin}
\end{frame}

\begin{frame}
  \frametitle{WHILE programs}
  \Large
  \begin{changemargin}{2em}
    A WHILE program is an \texttt{slist}, \\
    which is one or more statements. \\[1em]

    We need to prove the proposition for statements.\\[1em]

    Statements may contain
    arithmetic or boolean expressions.
  \end{changemargin}
\end{frame}

\begin{frame}
  \frametitle{Subgoal}
  \Large
  \begin{changemargin}{2em}
\emph{Lemma.}
  Evaluation of any boolean or arithmetic expression always yields a value and terminates.
  \end{changemargin}
\end{frame}


\begin{frame}
  \frametitle{Proof of Lemma}
  \Large
  \begin{changemargin}{2em}
    Start with arithmetic expressions.\\[1em]

    Base case: integers $n$ and variables $x$. From the semantics we have rules:
    
\[
\begin{array}{ccc}
  \begin{prooftree}
  \infer0{\langle n, q \rangle \Downarrow n},
\end{prooftree} & \hspace*{3em} & 
\begin{prooftree}
  \infer0{\langle x, q \rangle \Downarrow q(x)}
\end{prooftree}
\end{array}
\]
which clearly yield values $n$ and $q(x)$ \& terminate.

  \end{changemargin}
\end{frame}

\begin{frame}
  \frametitle{Inductive cases for lemma: binary}
  \begin{changemargin}{2em}
    Rules: negation, parentheses, arithmetic.\\[1em]

    Inductively assume all smaller expressions yield values \& terminate.\\[1em]

    Let's see $+$.
\[
  \begin{prooftree}
    \hypo{\langle e_1, q \rangle \Downarrow n_1}
    \hypo{\langle e_2, q \rangle \Downarrow n_2}
  \infer2{\langle e_1 + e_2, q \rangle \Downarrow n_1 + n_2}
  \end{prooftree}
  .
  \]
  By induction, $e_1$ and $e_2$ have the property.\\[1em]
  This derivation tree shows: if you build an expr wih $+$, it also has
  the desired property.\\[1em]

  Can say ``similarly for + and *''.
  \end{changemargin}
\end{frame}

\begin{frame}
  \frametitle{Inductive cases for lemma: unary}
  \begin{changemargin}{2em}
    Let's do $(e)$. Assume property holds for $e$. Then:
\[
  \begin{prooftree}
    \hypo{\langle e, q \rangle \Downarrow n}
  \infer1{\langle (e), q \rangle \Downarrow n},
  \end{prooftree}
.
\]
Can conclude that $(e)$ also yields a value and terminates.\\
Unary negation is the same.
  \end{changemargin}
\end{frame}

\begin{frame}
  \frametitle{Boolean Expressions}
  \Large
  \begin{changemargin}{2em}
    For these expressions, base cases are \texttt{true} and \texttt{false}.\\[1em]
    Quote the inference
rules
\[
\begin{array}{ccc}
  \begin{prooftree}
  \infer0{\langle \texttt{true}, q \rangle \Downarrow \texttt{true}},
\end{prooftree} & \hspace*{2em} &
\begin{prooftree}
  \infer0{\langle \texttt{false}, q \rangle \Downarrow \texttt{false}}
\end{prooftree}
\end{array}
\]
to conclude termination.

  \end{changemargin}
\end{frame}

\begin{frame}
  \frametitle{Boolean Expressions: Relational Operators}
  \Large
  \begin{changemargin}{2em}
    For $<$, $\le$, etc, we rely on termination for arithmetic expressions.
\[
  \begin{prooftree}
    \hypo{\langle e_1, q \rangle \Downarrow n_1}
    \hypo{\langle e_2, q \rangle \Downarrow n_2}
  \infer2{\langle e_1 < e_2, q \rangle \Downarrow (n_1 < n_2)}
  \end{prooftree}
  .
\]
$e_1$ and $e_2$ evaluate to integers, and then we apply the rule and return
\texttt{true} if $n_1 < n_2$ and \texttt{false} otherwise.
  \end{changemargin}
\end{frame}

\begin{frame}
  \frametitle{Boolean Expressions: Boolean Operators}
  \Large
  \begin{changemargin}{2em}
    For \texttt{and} and \texttt{or}, we rely on the induction hypothesis.
\[
  \begin{prooftree}
    \hypo{\langle e_1, q \rangle \Downarrow b_1}
    \hypo{\langle e_2, q \rangle \Downarrow b_2}
  \infer2{\langle b_1 \texttt{~and~} b_2, q \rangle \Downarrow (b_1 \wedge b_2)}
  \end{prooftree}
  .
\]
By IH, $e_1$ and $e_2$ evaluate to $b_1$ and $b_2$.\\
The quoted rule
yields a value for $b_1 \texttt{~and~} b_2$.\\[1em]

You should also mention \texttt{not} and the parenthesized $(e)$ here.~~\qedsymbol
  \end{changemargin}
\end{frame}

\begin{frame}
  \frametitle{Back to Statements}
  \Large
  \begin{changemargin}{2em}
The base cases for this proof are \texttt{skip}, assignment statements, and the \texttt{print\_state},
\texttt{assert}, \texttt{assume}, and \texttt{havoc} statements. \\[1em]
We only gave semantics for \texttt{skip} and
assignment, so we'll not talk about the other statements here either.
  \end{changemargin}
\end{frame}

\begin{frame}
  \frametitle{Base Case: Assignment Statement}
  \Large
  \begin{changemargin}{2em}
\[
  \begin{prooftree}
    \hypo{\langle e, q \rangle \Downarrow n}
  \infer1{\langle x := e, q \rangle \Downarrow q[x:=n]}
  \end{prooftree}
  .
\]

This is a base case: no statements in the hypothesis.\\[1em]

Per lemma, $e$ evals to value $n$ and terminates. \\[1em]

This rule shows that we can evaluate an
assignment statement and terminate. \\
(Big-step semantics: evaluation just changes the state.)
  \end{changemargin}
\end{frame}

\begin{frame}
  \frametitle{Inductive case: if Statement}
  
  \Large
  \begin{changemargin}{2em}
Termination because we inductively assume termination for then and else clauses.

\[
  \begin{prooftree}
    \hypo{\langle s_1, q \rangle \Downarrow q'}
    \hypo{\langle e, q \rangle \Downarrow \mbox{true}}
  \infer2{\langle \texttt{if $e$ then $s_1$ else $s_2$}, q \rangle \Downarrow q'}
  \end{prooftree}
  .
  \]
  $s_1$ terminates by IH
  and so the \texttt{if} also terminates by this derivation.\\[1em]
    ``Similiarly for \texttt{else}''.


  \end{changemargin}
\end{frame}

\begin{frame}
  \frametitle{Inductive case: statement list}
  
  \Large
  \begin{changemargin}{2em}
\[
  \begin{prooftree}
    \hypo{\langle s_1, q \rangle \Downarrow q^1}
    \hypo{\cdots}
    \hypo{\langle s_n, q^{n-1} \rangle \Downarrow q^n}
  \infer3{\langle \{ s_1 ; \cdot ;s s_n \}, q \rangle \Downarrow q^n}
  \end{prooftree}
  .
  \]
Because, inductively, all of the $s_i$ terminate, then so does the list of $s$. ~~\qedsymbol
  \end{changemargin}
\end{frame}

  \end{document}

