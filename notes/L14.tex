\documentclass[11pt]{article}
\usepackage{mathpartir}
\usepackage{pifont}
\usepackage{multicol}
\usepackage{listings}
\usepackage{pgf}
\usepackage{tikz}
\usepackage{alltt}
\usepackage{hyperref}
\usepackage{url}
\usepackage{amsmath}
\usepackage{amssymb}
\usepackage{dafny}
\usetikzlibrary{arrows,automata,shapes,positioning}
\tikzstyle{block} = [rectangle, draw, fill=blue!20, 
    text width=5em, text centered, rounded corners, minimum height=2em]
\tikzstyle{bt} = [rectangle, draw, fill=blue!20, 
    text width=1em, text centered, rounded corners, minimum height=2em]
\newcommand{\xmark}{\ding{55}}

\newtheorem{defn}{Definition}
\newtheorem{crit}{Criterion}
\newcommand{\true}{\mbox{\sf true}}
\newcommand{\false}{\mbox{\sf false}}

\newcommand*\circled[1]{\tikz[baseline=(char.base)]{
            \node[shape=circle,draw,inner sep=2pt] (char) {#1};}}


\newcommand{\handout}[5]{
  \noindent
  \begin{center}
  \framebox{
    \vbox{
      \hbox to 5.78in { {\bf Software Testing, Quality Assurance and Maintenance } \hfill #2 }
      \vspace{4mm}
      \hbox to 5.78in { {\Large \hfill #5  \hfill} }
      \vspace{2mm}
      \hbox to 5.78in { {\em #3 \hfill #4} }
    }
  }
  \end{center}
  \vspace*{4mm}
}

\newcommand{\lecture}[4]{\handout{#1}{#2}{#3}{#4}{Lecture #1}}
\topmargin 0pt
\advance \topmargin by -\headheight
\advance \topmargin by -\headsep
\textheight 8.9in
\oddsidemargin 0pt
\evensidemargin \oddsidemargin
\marginparwidth 0.5in
\textwidth 6.5in

\parindent 0in
\parskip 1.5ex
%\renewcommand{\baselinestretch}{1.25}

\usepackage{enumitem}

\newtheorem{prop}{Proposition}
\newtheorem{lemma}{Lemma}
\usepackage{ebproof}
\newcommand{\qedsymbol}{\rule{1ex}{1ex}}
\newcommand{\sem}[3]{\langle #1, #2 \rangle \Downarrow #3}

\lstset{ %
language=Java,
basicstyle=\ttfamily,commentstyle=\scriptsize\itshape,showstringspaces=false,breaklines=true,numbers=left}

%\usepackage{fontspec}
%\setmonofont{Cousine}[Scale=MatchLowercase]

\begin{document}

\lecture{14 --- March 10, 2025}{Winter 2025}{Patrick Lam}{version 1}

Other applications of verified software besides Cedar:
\begin{itemize}[noitemsep]
\item Paris Metro line 14 control system (B)\footnote{\url{https://www.clearsy.com/wp-content/uploads/2020/03/Formal-methods-for-Railways-brochure-mai-2020.pdf}}: 110,000 lines of models, 86,000 lines of Ada.
\item seL4 (Haskell, Isabelle/HOL, C)\footnote{\url{https://sel4.systems/Info/FAQ/proof.html}}: the seL4 microkernel conforms to its specification and maintains integrity and confidentiality.
\item CompCert (Coq)\footnote{https://inria.hal.science/hal-01238879}: a formally verified optimizing compiler
\item IronClad (Dafny)\footnote{https://www.microsoft.com/en-us/research/project/ironclad/}: guarantees about how remote apps behave
\end{itemize}

The first three applications were carried out by verification experts; the last one was done by systems programmers. In all of these cases, the systems were designed as verified systems from the start.

\section*{Dafny}

I don't know how long it will take, but in this set of lecture notes, we'll aim to get through
the first online guide, which shows how to verify basic imperative code (including loops but not advanced
topics like objects, sequences and sets, data structures, lemmas). I'm going to summarize what's
there, and in class, we'll work on exercises together.

\begin{itemize}[noitemsep]
\item \url{https://dafny.org/dafny/OnlineTutorial/guide}
\item \url{https://dafny.org/blog/2023/12/15/teaching-program-verification-in-dafny-at-amazon/}
\end{itemize}

\section*{Dafny: annotations}
This annotation in Dafny should be familiar to you from what we've seen in class:
\begin{lstlisting}[language=dafny]
  forall k: int :: 0 <= k < a.Length ==> 0 < a[k]
\end{lstlisting}
or in ASCII,
\begin{verbatim}
  forall k: int :: 0 <= k < a.Length ==> 0 < a[k]
\end{verbatim}
It says that all elements of array $a$ are positive.

Dafny proves user-specified annotations and verifies the absence of runtime errors---the things
that are usually undefined behaviour or crashes, like out of bounds array indexes, null dereferences, etc.
We've seen how to use symbolic execution to avoid these as well, but Dafny is a fully-static approach
that aims to work on real code (written in the Dafny language).

\end{document}
